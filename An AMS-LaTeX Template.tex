\documentclass[11pt,letterpaper,pdftex]{amsart} 

%Always declare the document class at the start of the document. In this case, we are using the AMS article style. AMS stands for American Mathematical society.
%The 11 inside the square brackets determines the font size. If no number is declared, e.g., if one types  \documentclass{amsart}, then font size defaults to 10. 

%All of the above text, and any text below before the \begin{document} tag is part of the *the preamble*, where packages are loaded, commands and shortcuts
%created, code for generating advanced page formatting is inserted, etc. 
%For example, here are some packages.  Many of the uncommented ones enable the typesetting of certain symbols that aren't directly available with the basic
%AMS article class.

\usepackage{amsmath} %Allows access to a wide array of math symbols widely used in AMS-LaTeX
\usepackage{amstext} %contains methods for including text inside of math environments, including in sub/superscripts
\usepackage{amscd} %used for commutative diagrams; more sophisticated alternatives exist, like tikz-cd
\usepackage{amsthm} %needed for theorem environments
\usepackage{amssymb} %Includes useful additional symbols not found in the amsmath package
\usepackage{mathrsfs} %Supports the commonly used math script font
\usepackage{multicol} %For making tunable environments for multiple column lists
\usepackage{verbatim} %Needed for easy quoting of LaTeX code not to be executed during compiling, but displayed "verbatim" instead
\usepackage{yhmath} %More fonts?
\usepackage{dsfont} %Googling a LaTeX package name allows one to learn what the package contains, find readmes, download the latest version, etc.
\usepackage[pdftex]{graphics}
%\usepackage{xcolor} %uncomment to use. xcolor can be used to specify text colorings.
\usepackage{geometry}
%\usepackage{soul}
\geometry{
  top=0.75in,
  inner=1in,
  outer=1in,
  bottom=0.75in,
  headheight=3ex,
  headsep=2ex,
}
%The geometry command above specifies settings for the page geometry, and controls margins.

%The following three packages have to be loaded in the specified order in order to prevent option clashes.
\usepackage{graphicx}
\usepackage[dvipsnames]{xcolor} %this is xcolor with the dvipsnames option. 
\usepackage{epstopdf} %Converts EPS vector graphics into PDF images for inclusion in output .pdf


% Theorem environment setup
%Note that two versions of each theorem environment are defined: one which associates a section number label, and one which does not.
\theoremstyle{definition}
\newtheorem*{dfn*}{Definition} 
\newtheorem{dfn}{Definition}[section]

\newtheorem*{rmk*}{Remark}
\newtheorem{rmk}{Remark}[section]

\newtheorem*{obv*}{Observation}
\newtheorem{obv}{Observation}[section]

\newtheorem*{clm*}{Claim}
\newtheorem{clm}{Claim}[section]

\newtheorem*{ex*}{Example}
\newtheorem{ex}{Example}[subsection] %note that examples will number by subsection instead of section by this code

\newtheorem*{exc*}{Exercise}
\newtheorem{exc}{Exercise}[subsection] %similarly for exercises re: subsection numbering

\theoremstyle{plain}
\newtheorem*{thm*}{Theorem}
\newtheorem{thm}{Theorem}[section]

\newtheorem*{cor*}{Corollary}
\newtheorem{cor}{Corollary}[section]

\newtheorem*{cnj*}{Conjecture}
\newtheorem{cnj}{Conjecture}[section]

\newtheorem*{lem*}{Lemma}
\newtheorem{lem}{Lemma}[section]

\newtheorem*{prp*}{Proposition}
\newtheorem{prp}{Proposition}[section]

%Often, one would create their own style file, e.g., ahavens.sty, to contain commands that might be common across your workflow/used in many preambles.  
%For example one would probably opt to include the above theorem environment set up commands in a style file, to shorten the preamble and make the document %easier to parse. To load a style file, one adds, e.g., \usepackage{ahavens} replacing the name "ahavens" with the name of your own style file.

%Here we can add shortcut commands.  For example, since we may often want the blackboard bold R for the set of all real numbers, we may create a command:
\newcommand{\RR}{\mathbb{R}}
%Here are some more such commands that I commonly use; feel free to delete, modify, and craft your own suited to your needs:
\newcommand{\ud}{\mathrm{d}} %upright d
\newcommand{\dif}{\, \ud} %upright d for differentials
\newcommand{\CC}{\mathbb{C}} %Complex Numbers
\newcommand{\HH}{\mathbb{H}} %Quaternions
\newcommand{\ddt}{\tfrac{\ud}{\ud t}} %d/dt operator, 
\renewcommand{\v}[1]{\mathbf{#1}} %makes a bold symbol for vectors, used e.g. as \v{a} to make a bold a
\renewcommand{\u}[1]{\hat{\v{#1}}} %Bolds a letter and puts a hat on it, to denote a unit vector
\newcommand{\ddv}[1]{\frac{\ud}{\ud #1}} %makes a derivative operator with respect to a specified variable, e.g., d/dx is generated by \ddv{x} 
\newcommand{\dfndvr}[2]{\frac{\ud #1}{\ud #2}} %another derivative e.g., to make df/dx, instead of typing \frac{\ud g}{\ud t}, simply type dfndvr{g}{t}
\newcommand{\pr}[2]{\operatorname{proj}_{\v{#1}}\v{#2}} %various shortcuts for the vector projection operator.  Why so many? Figure it out!
\newcommand{\pru}[2]{\operatorname{proj}_{\u{#1}}\v{#2}}
\newcommand{\pruu}[2]{\operatorname{proj}_{\u{#1}}\u{#2}}
\newcommand{\prru}[2]{\operatorname{proj}_{\v{#1}}\u{#2}}
\newcommand{\cmp}[2]{\operatorname{comp}_{\v{#1}}\v{#2}}
\newcommand{\pd}[2]{\frac{\partial #1}{\partial #2}} %And now partial derivative shortcuts...
\newcommand{\pdvr}[1]{\frac{\partial}{\partial #1}}
\newcommand{\pdv}[1]{\partial_{#1}}
\newcommand{\ppd}[3]{\frac{\partial^2 #1}{\partial #2 \, \partial #3}}
\newcommand{\pdd}[2]{\frac{\partial^2 #1}{\partial #2^2}}

%We're almost ready to start the document! Just need to get the title info ready.

\title{An \AmS-\LaTeX\ Template}
\author{A. Havens}
\address{Department of Mathematics and Statistics, University of Massachusetts, Amherst} 
%If there is a coauthor, uncomment the code below.
%\author{Second Author}
%\address{Department of Mathematics and Statistics, University of Massachusetts, Amherst}
\date{\today}
%You should change these details!

%The next command is crucial: to end the preamble, you will have to *begin the document* !!! You also must remember to end it.
%% This is the end of the preamble.
\begin{document}
%Here's how to instantiate the abstract environment, where a summary of an articles premise and and important content is given.
\begin{abstract}
Here is where you would type an abstract, if you have/need one.
\end{abstract}
%You probably don't need to include an abstract every time, but sometimes you will. 
%For AMS articles you always have to place the abstract before the \maketitle command.
\maketitle
%without the above line, there will be no title!

%You may consider deleting everything between this line and the line above the \end{document} command when you begin to use this in earnest. 
%Just remember how to instantiate the needed environments!

\section{A section.}
\subsection{A subsection.}
%This is where the main body text goes.
Some body text. I can write a paragraph. Blah blah blah blah

\bigskip %big skip makes a space between paragraphs.

\hrule % this makes a horizontal line

\vfill %I inserted \vfills between sections to make the white space fill in between sections, instead of at the page bottom.
%\vfill is most useful for making the space between numbered exercises.

\section{A New Section.}

%To instantiate a theorem environment, you again use \begin{...} ... \end{...}, specifying your preferred type of environment.
\begin{thm}[A false theorem] Every theorem is false.
\end{thm}

%And here's how to instantiate a proof environment:
\begin{proof} If a theorem were true, then I'd know about it.  Since I know of no true theorem, they all must be false, including the above theorem.
\end{proof}

\begin{thm*}[A true theorem] The ``theorem'' above asserts its own falsehood, and thus is self-contradictory, and not a theorem.
\end{thm*}
%note that the first theorem environment yields a number, and the new one with the asterisk does not.

\begin{proof} \textcolor{RoyalPurple}{The proof is an exercise best left to the reader in this case.}
\end{proof}

\vfill

\section{A Final Section}

\subsection{A few more environments}
%A definition, showcasing italic/emphasized text, inline math, and a display math environment.
\begin{dfn} A \emph{quaternion} is a real linear combination of the basis elements $1$, $i$, $j$, and $k$, where $i$, $j$, and $k$, satisfy the relations
\[i^2=j^2=k^2=ijk=-1 \, .\]
The space of all quaternions, denoted $\HH$, is a $4$ dimensional vector space and a real normed division algebra with non-commutative multiplication.
\end{dfn}

I want to create an integral on this line: %$\displaystyle{\int_{\RR} e^{-x^2} \dif x}$

\begin{rmk} A remark.
\end{rmk}

\begin{clm*} A claim, unnumbered.
\end{clm*}

\begin{lem} A lemma.
\end{lem}

\begin{cnj*} A conjecture, unnumbered.
\end{cnj*}

\subsection{A final subsection}

\begin{prp} \color{red}A proposition in red.%observe how the red continues throughout the environment.
\end{prp}

\begin{ex} An example, numbered by subsection, per the preamble. 
\end{ex}

\begin{exc} An exercise, also numbered by subsection.
\end{exc}

\noindent A numbered list:
\begin{enumerate} 
\item the first item
\item the second item 
\item the third and final item in the document.
\end{enumerate} %and we close all environments,
\end{document} %including the document itself!